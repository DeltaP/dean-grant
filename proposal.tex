\documentclass[11pt]{article}
\usepackage[margin=1.0in]{geometry}
\usepackage{sectsty}
\sectionfont{\large}

\begin{document}
  \title{GPU's for Lattice Gauge Theory}
  \author{Anqi Cheng}
  \maketitle

  \section*{Overview} %although the layout may seem awkward this is the paragraph that they say should have all the information....
  In particle physics, lattice gauge theory is an important branch to connect experimental data and physical models.
  It is extremely computationally demanding and therefore should take advantage of new  resources as they become available.
  One emerging resource that I am eager to break into is Graphics Processing Units (GPU's).   
  My proposal is to build a workstation around two of NVIDIA's latest GPU's, the Kepler K20.  
  This workstation will be used for GPU code development and data analysis, which will benefit the doctoral dissertation of myself as well as another 
  graduate student in our group, Gregory Petropoulos. 
  To accomplish our goal we are requesting \$10,000.  
  
  \section*{Code Development aka Coding for the Future} % emphisize the need of the resource for dedicated code development, the importance of having two cards...
  The rapid growth of programmability and computation power of GPUs has trigged transformational effects in many scientific applications. 
  As GPUs become a very attractive platform for lattice gauge theory simulations, related GPU libraries (e.g. QUDA \cite{QUDA1,QUDA2,QUDA3}) have been developed and provide a powerful framework for lattice physicists to exploit.
  Meanwhile many large GPU clusters have been built, e.g., Edge cluster at LLNL, Dsg cluster at FNAL, and 10g cluster at JLab.

  On the one hand, my research involves large scale computations which can only be performed on publicly available supercomputing resources. 
  Having a local GPU machine for code development and testing could enable our group apply for computer time on those GPU clusters.
  On the other hand, a local GPU machine could potentially dramatically speed up our local data analysis process.
  Therefore building the GPU workstation could greatly benefit my research and doctoral dissertation, as well as the future development of our group.
 
  \section*{Methods} % describe the lattice workflow
  Lattice QCD workflow is generally broken into three major stages:  generate configurations, measure observables, analyze data.
  GPU's are generally used most in the first two stages.  
  To meet our needs

  
  \section*{Its all about the Flops} % when code development is over this will be a valuable compute resource
  When not being used for code development the machine will serve as a dedicated computational resource.  
  The machine that I propose has two six core processors and two Keperl GPU's and can provide a theoretical peak of 2.5 Teraflops (trillion floating point operations per second).  
  As a dedicated resource that we have access to this is 

  \section*{Qualifications} %classic brag sheet
  Traditional coding is a basic skill in my research. 
  Besides I took a high performance scientific computation class from the Computer Science Department last semester, by which I got familiar with large-scale parallel computation skills. 
  Furthermore I attend the lectures on GPUs given by Michael Clark, one of the founders of GPU computing, both on USQCD All Hands' Meeting at Fermilab last April and at INT summer program held by University of Washington last August. 
  From the lectures I learned the basis of GPU computing and its recent development in my research field. With the knowledge and skills I mentioned above I am confident to successfully accomplish this project.

% ---- Bibliography ----
%
\begin{thebibliography}{99}
\bibitem{QUDA1}
http://lattice.github.com/quda/
\bibitem{QUDA2}
M. A. Clark, R. Babich, K. Barros, R. C. Brower, and C. Rebbi,
``Solving Lattice QCD systems of equations using mixed precision solvers on GPUs" Comput. Phys. Commun., vol. 181, 2010, p. 1517. [arXiv:0911.3191 
[hep-lat]]
\bibitem{QUDA3}
R. Babich, M. A. Clark, B. Jo$\acute{o}$ ``Parallelizing the QUDA Library for Multi-GPU Calculations in Lattice Quantum Chromodynamics" [arXiv:1011.0024 
[hep-lat]]

\end{thebibliography}

\end{document}
