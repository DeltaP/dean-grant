\documentclass[11pt]{article}
\usepackage[margin=1.0in]{geometry}

\begin{document}
  \title{GPU's for Lattice Field Theory}
  \author{Anqi Cheng}
  \maketitle

   \section*{Summary} %although the layout may seem awkward this is the paragraph that they say should have all the information....
In particle physics, lattice gauge theory (LGT) is an important branch to connect experimental data and physical models.
It is extremely computationally demanding and therefore should take advantage of new  resources as they become available.
One emerging resource that I am eager to break into is Graphics Processing Units (GPU's).   
My proposal is to build a workstation around two of NVIDIA's latest GPU's, the Kepler K20.  
This workstation will be used for GPU code development and data analysis, which will benefit the doctoral dissertation of myself 
as well as another graduate student, Gregory Petropoulos.  
To accomplish our goal we are requesting \$10,000.  
 
  \section*{Maping out the Frontier of Physics} % brief overview of our reserach, standard model, strong dynamics, bsm, lattice...
  
  
  \section*{Code Development aka Coding for the Future} % emphisize the need of the resource for dedicated code development, the importance of having two cards...
 The rapid growth of programmability and computation power of GPUs has trigged transformational effects in many scientific applications. As GPUs becomes 
 a very attractive platform for LGT simulations, related GPU libraries (e.g. QUDA) have been developed and  
 provide a powerful framework for lattice physicists to exploit. Meanwhile many large GPU clusters have been built, e.g., Edge cluster at LLNL,
 Dsg cluster at FNAL, and 10g cluster at JLab.\\\\
 My research involves large scale computations which can only be performed on publicly available supercomputing resources. WIth the lack of a local GPU machine for code development and testing, our group is excluded from the GPU code development in our research field. 
 To take advantage of those off campus GPU clusters it is critical to have a local GPU machine on which we can develop and test code. 
 Once we have working GPU code, the number of resources which we can apply for time on will increase substantially.\\\\

 The machine proposed in this grant would fill this gap and allow us to participate in GPU code development.  One aspect of the machine that I have proposed that is important is that there are two Kepler cards.  Many of the new clusters have hundreds of similar cards.  
  
  \section*{Its all about the Flops} % when code development is over this will be a valuable compute resource
  When not being used for code development the machine will serve as a dedicated computational resource.  
  The machine that I propose has two six core processors and two Keperl GPU's and can provide a theoretical peak of 2.5 Teraflops (trillion floating point operations per second).  
  As a dedicated resource that we have access to this is

  \section*{What this Project Will Support} %specifics of what this will support?
  This grant will support my own thesis work studying the eigenvalues of str

  \section*{Qualifications} %classic brag sheet


\end{document}
